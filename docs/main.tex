

\documentclass[10pt]{article}

\usepackage[utf8]{inputenc}

 \usepackage[T1]{fontenc}
\usepackage{xcolor}
\usepackage{float}
\usepackage{graphicx}
\usepackage{siunitx}
\usepackage{geometry}
\usepackage{float}
\usepackage[toc,page]{appendix}
\graphicspath{ {./figuren/} }
\parindent=0pt
\usepackage{fancyvrb}
\usepackage{hyperref}
\hypersetup{
    colorlinks = true,
    linkcolor = blue,
    citecolor=blue,
    linkbordercolor = {white},
}
\usepackage{fullpage}
\usepackage{fancyvrb}

\frenchspacing
\usepackage{booktabs}
\usepackage{microtype}

\usepackage[english,dutch]{babel}

\usepackage{listings}
% Er zijn talloze parameters ...
\lstset{
    belowcaptionskip=1\baselineskip,
    breaklines=true,
    columns=flexible,
    basicstyle=\small\ttfamily,
}
\usepackage{graphicx}
\usepackage{placeins}

\usepackage{mathtools}
\setlength{\parskip}{\baselineskip}%
\setlength{\parindent}{1pt}%

\title{Opdracht 1, Algoritmiek}
\author{Jenny Vermeltfoort, s3787494, groep PO1\_20}

\begin{document}

\selectlanguage{dutch}
\def\tablename{Tabel}

\maketitle

\section{Uitleg}


\subsection{Optimale score.}


\section{Puzzel oplossingen}

\begin{table}[h]
    \centering
    \caption{Resultaten van verschillende puzzels.}
    \begin{tabular}{@{}ccccc@{}}
        \toprule
        Puzzel & Grondgetal & Oplossingen & Bekeken & Tijdsduur \\ 
        \midrule
        STOOM+BOOT=PAKJE & 10 & 5      & 1099 & \SI{29}{\micro\second}. \\
        SINT+PIET=RUZIE & 9 & -1       & 229 & \SI{5}{\micro\second}. \\
        SINT+PIET=RUZIE & 10 & 16      & 397 & \SI{14}{\micro\second}. \\
        SINT+PIET=RUZIE & 11 & 64      & 859 & \SI{21}{\micro\second}. \\
        SINT+PIET=RUZIE & 20 & 11068   & 40482 & \SI{883}{\micro\second}. \\
        SINT+PIET=RUZIE & 26 & 53590   & 162415 & \SI{3264}{\micro\second}. \\
        SINT+PIET=RUZIE, N=3, P=21 & 26 & 213 & 733  & \SI{13}{\micro\second}. \\
        \bottomrule
    \end{tabular}
\end{table}
\FloatBarrier


\section{Puzzel constructies}

\begin{table}[h]
    \centering
    \caption{Resultaten van verschillende constructies.}
    \begin{tabular}{@{}cccc@{}}
        \toprule
        Woorden & Grondgetal & Puzzels & Tijdsduur \\ 
        \midrule
        SINT+PIET=... & 10 & 840 & \SI{0.184}{\second}. \\
        KLAAS+PAARD=... & 10 & 24864 & \SI{6.074}{\second}. \\
        MIJTER+MANTEL=... & 10 & 461320 & \SI{103.948}{\second}. \\
        \bottomrule
    \end{tabular}
\end{table}
\FloatBarrier




\begin{appendices}
    \section{Experimenten Logs}
    \begin{lstlisting}

        De puzzel kent 5 verschillende oplossing(en).
        Het zoeken van de oplossingen kostte 28 clock ticks, ofwel 2.8e-05 seconden.
        We hebben daarbij 1099 deeloplossingen bekeken.
        Een gevonden oplossing is:
        M=2 T=7 O=5 B=8 S=3 E=9 J=0 K=1 A=6 P=4
        Grontgetal: 10 
        Puzzel:
        0: STOOM
        1: BOOT
        2: PAKJE
        Toegekende letters:

        De puzzel kent -1 verschillende oplossing(en).
        Het zoeken van de oplossingen kostte 5 clock ticks, ofwel 5e-06 seconden.
        We hebben daarbij 229 deeloplossingen bekeken.
        Grontgetal: 9 
        Puzzel:
        0: SINT
        1: PIET
        2: RUZIE
        Toegekende letters:
        R=1; 

        De puzzel kent 16 verschillende oplossing(en).
        Het zoeken van de oplossingen kostte 14 clock ticks, ofwel 1.4e-05 seconden.
        We hebben daarbij 397 deeloplossingen bekeken.
        Een gevonden oplossing is:
        T=2 N=5 E=4 I=9 S=6 P=3 Z=8 U=0 R=1
        Grontgetal: 10 
        Puzzel:
        0: SINT
        1: PIET
        2: RUZIE
        Toegekende letters:
        R=1; 

        De puzzel kent 64 verschillende oplossing(en).
        Het zoeken van de oplossingen kostte 21 clock ticks, ofwel 2.1e-05 seconden.
        We hebben daarbij 859 deeloplossingen bekeken.
        Een gevonden oplossing is:
        T=2 N=6 E=4 I=10 S=7 P=8 Z=9 U=5 R=1
        Grontgetal: 11 
        Puzzel:
        0: SINT
        1: PIET
        2: RUZIE
        Toegekende letters:
        R=1; 

        De puzzel kent 11068 verschillende oplossing(en).
        Het zoeken van de oplossingen kostte 883 clock ticks, ofwel 0.000883 seconden.
        We hebben daarbij 40482 deeloplossingen bekeken.
        Een gevonden oplossing is:
        T=2 N=3 E=4 I=7 S=5 P=15 Z=14 U=0 R=1
        Grontgetal: 20 
        Puzzel:
        0: SINT
        1: PIET
        2: RUZIE
        Toegekende letters:
        R=1; 

        De puzzel kent 53590 verschillende oplossing(en).
        Het zoeken van de oplossingen kostte 3264 clock ticks, ofwel 0.003264 seconden.
        We hebben daarbij 162415 deeloplossingen bekeken.
        Een gevonden oplossing is:
        T=2 N=3 E=4 I=7 S=5 P=21 Z=14 U=0 R=1
        Grontgetal: 26 
        Puzzel:
        0: SINT
        1: PIET
        2: RUZIE
        Toegekende letters:
        R=1; 

        De puzzel kent 213 verschillende oplossing(en).
        Het zoeken van de oplossingen kostte 13 clock ticks, ofwel 1.3e-05 seconden.
        We hebben daarbij 733 deeloplossingen bekeken.
        Een gevonden oplossing is:
        T=2 N=3 E=4 I=7 S=5 P=21 Z=14 U=0 R=1
        Grontgetal: 26 
        Puzzel:
        0: SINT
        1: PIET
        2: RUZIE
        Toegekende letters:
        N=3; P=21; R=1; 

        # echo "2 10 SINT PIET 3 4 3" | ./WoordSomPuzzel
        We vonden 840 puzzels met een unieke oplossing.
        Het construeren van de puzzels kostte 184744 clock ticks, ofwel 0.184744 seconden.
        Een mogelijk woord 2 met een unieke oplossing: TTNSN

        # echo "2 10 KLAAS PAARD 3 4 3" | ./WoordSomPuzzel
        We vonden 24864 puzzels met een unieke oplossing.
        Het construeren van de puzzels kostte 6074723 clock ticks, ofwel 6.07472 seconden.
        Een mogelijk woord 2 met een unieke oplossing: SSSDDK

        # echo "2 10 MIJTER MANTEL 3 4 3" | ./WoordSomPuzzel 
        We vonden 461320 puzzels met een unieke oplossing.
        Het construeren van de puzzels kostte 103947895 clock ticks, ofwel 103.948 seconden.
        Een mogelijk woord 2 met een unieke oplossing: RRRREIJ

        
    \end{lstlisting}
 

\end{appendices}

\end{document}